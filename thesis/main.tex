\documentclass[english, inz, shortabstract]{iithesis}
\usepackage[utf8]{inputenc}
\usepackage{hyperref}
\usepackage{dirtytalk}
\usepackage{listings}
\usepackage{graphicx}

\author{Mikołaj Depta}
\advisor{dr Andrzej Łukaszewski}
\transcriptnum {328690}
\advisorgen    {dr. Andrzeja Łukaszewskiego}
\englishtitle{Analysis of Type--Driven approach to systems programming:\fmlinebreak Implementation of OpenGL library for Rust}

\polishtitle{Analiza programowania systemowego z wykorzystaniem systemu typów:\fmlinebreak Implementacja biblioteki do OpenGL dla języka Rust}

\englishabstract{For the past few years in the software development industry there has been a growing interest in strongly typed languages. 
It manifests itself in emergence of brand--new technologies in which strong type systems were one of the core founding principles or in changes introduced to existing languages. 
The most common examples of modern languages with powerful type systems are TypeScript as an alternative to JavaScript in the world of web development or Rust in domain of systems programming in place of C and C++. 
More mature languages also had their type systems revised for example in C\# 8 explicit type nullability annotations were introduced, 
or even dynamically typed Python has seen major improvements to its type annotation system.

This study --- an implementation of the OpenGL graphics API wrapper library for Rust --- will attempt to demonstrate 
how Rust's type system can be utilized to improve low--level software safety and maintainability as well as how it affects API design and codebase structure.
}

\polishabstract{Od kilku lat w branży programistycznej rośnie zainteresowanie językami silnie typowanymi. Przejawia się to przez powstawanie zupełnie nowych technologii, 
w których silne systemy typów są fundamentem budowy języka lub w zmianach wprowadzanych do istniejących języków. Przykładami takich nowoczesnych języków są TypeScript, 
jako alternatywa dla JavaScript w świecie stron internetowych lub Rust w dziedzinie programowania systemowego zamiast C i C++. Starsze technologie rozszerzyły swoje systemy typów, 
na przykład w C\# 8 wprowadzono jawne adnotacje o tym, że typ może przyjmować wartości \texttt{null}. 
Nawet dynamicznie typowany Python doczekał się znacznych ulepszeń w swoim systemie adnotacji typów \cite{pythontyping}.

Niniejsza praca --- implementacja biblioteki opakowującej API OpenGL dla języka Rust --- będzie próbą zademonstrowania w jaki sposób jego 
system typów może zostać wykorzystany do zwiększenia niezadwodności, łatwości utrzymania oraz rozwoju oprogramowania i 
jego niskopoziomowego bezpieczeństwa; a także w jaki sposób wpływa on na projektowanie API i strukturę kodu.
}

\begin{document}

\chapter*{Introduction}

This study aims to demonstrate how extensive use of advanced type systems can change the way software is developed.
Advancements in type systems allow their users to express increasingly more complex invariants and contracts for their programs using logic rules, 
and enable type checkers to proof that all these invariants upheld at compile-time.

This is very desired as type systems not only offer extensive compile-time verification, but just as importantly logic based rules are substantially less susceptible to bugs
due to programming errors like regular validation code is.

For over a decade now we have now seen gradual shift toward both stronger typing and more of a functional approach to program specification in many languages.
with strong and expressive types systems as a foundation of this trend.

We chose to implement wrapper library for OpenGL using Rust due to several important factors.

Rust has: a state of the art type system comparable to Haskell's, 
is natively compiled language with very good support for C interoperability,
its ownership based model of resource lifetime management provides strong guarantees about when exactly destructor will be called,
borrow checker allows to better express how different pieces of program use different 

OpenGL is a very mature, cross-platform, well understood and widely supported graphics specification.
Its also arguably the best API for beginners in hardware accelerated graphics programming.
With all that being said its also quite dated, and its sequential model of execution does accurately represent the realities of graphics hardware.

We realized that Rust's type system can help to amend certain aspects of OpenGL which make it sometimes difficult to use.
We tried to emphasize important aspects of the API like: control flow or graphics pipeline and 
impose a structure for different APIs by generalizing them, as well as, using strongly typed procedures and types.
so users can focus on graphics programming and not have to worry about different OpenGL obscurities and verifying by hand that 
numerous contracts are upheld.

\chapter*{Introduction}

This study aims to demonstrate how extensive use of advanced type systems can change the way software is developed.
Advancements in type systems allow their users to express increasingly more complex invariants and contracts for their programs using logic rules, 
and enable type checkers to proof that all these invariants upheld at compile-time.

This is very desired as type systems not only offer extensive compile-time verification, but just as importantly logic based rules are substantially less susceptible to bugs
due to programming errors like regular validation code is.

For over a decade now we have now seen gradual shift toward both stronger typing and more of a functional approach to program specification in many languages.
with strong and expressive types systems as a foundation of this trend.

We chose to implement wrapper library for OpenGL using Rust due to several important factors.

Rust has: a state of the art type system comparable to Haskell's, 
is natively compiled language with very good support for C interoperability,
its ownership based model of resource lifetime management provides strong guarantees about when exactly destructor will be called,
borrow checker allows to better express how different pieces of program use different 

OpenGL is a very mature, cross-platform, well understood and widely supported graphics specification.
Its also arguably the best API for beginners in hardware accelerated graphics programming.
With all that being said its also quite dated, and its sequential model of execution does accurately represent the realities of graphics hardware.

We realized that Rust's type system can help to amend certain aspects of OpenGL which make it sometimes difficult to use.
We tried to emphasize important aspects of the API like: control flow or graphics pipeline and 
impose a structure for different APIs by generalizing them, as well as, using strongly typed procedures and types.
so users can focus on graphics programming and not have to worry about different OpenGL obscurities and verifying by hand that 
numerous contracts are upheld.

\chapter*{Introduction}

This study aims to demonstrate how extensive use of advanced type systems can change the way software is developed.
Advancements in type systems allow their users to express increasingly more complex invariants and contracts for their programs using logic rules, 
and enable type checkers to proof that all these invariants upheld at compile-time.

This is very desired as type systems not only offer extensive compile-time verification, but just as importantly logic based rules are substantially less susceptible to bugs
due to programming errors like regular validation code is.

For over a decade now we have now seen gradual shift toward both stronger typing and more of a functional approach to program specification in many languages.
with strong and expressive types systems as a foundation of this trend.

We chose to implement wrapper library for OpenGL using Rust due to several important factors.

Rust has: a state of the art type system comparable to Haskell's, 
is natively compiled language with very good support for C interoperability,
its ownership based model of resource lifetime management provides strong guarantees about when exactly destructor will be called,
borrow checker allows to better express how different pieces of program use different 

OpenGL is a very mature, cross-platform, well understood and widely supported graphics specification.
Its also arguably the best API for beginners in hardware accelerated graphics programming.
With all that being said its also quite dated, and its sequential model of execution does accurately represent the realities of graphics hardware.

We realized that Rust's type system can help to amend certain aspects of OpenGL which make it sometimes difficult to use.
We tried to emphasize important aspects of the API like: control flow or graphics pipeline and 
impose a structure for different APIs by generalizing them, as well as, using strongly typed procedures and types.
so users can focus on graphics programming and not have to worry about different OpenGL obscurities and verifying by hand that 
numerous contracts are upheld.

\chapter*{Introduction}

This study aims to demonstrate how extensive use of advanced type systems can change the way software is developed.
Advancements in type systems allow their users to express increasingly more complex invariants and contracts for their programs using logic rules, 
and enable type checkers to proof that all these invariants upheld at compile-time.

This is very desired as type systems not only offer extensive compile-time verification, but just as importantly logic based rules are substantially less susceptible to bugs
due to programming errors like regular validation code is.

For over a decade now we have now seen gradual shift toward both stronger typing and more of a functional approach to program specification in many languages.
with strong and expressive types systems as a foundation of this trend.

We chose to implement wrapper library for OpenGL using Rust due to several important factors.

Rust has: a state of the art type system comparable to Haskell's, 
is natively compiled language with very good support for C interoperability,
its ownership based model of resource lifetime management provides strong guarantees about when exactly destructor will be called,
borrow checker allows to better express how different pieces of program use different 

OpenGL is a very mature, cross-platform, well understood and widely supported graphics specification.
Its also arguably the best API for beginners in hardware accelerated graphics programming.
With all that being said its also quite dated, and its sequential model of execution does accurately represent the realities of graphics hardware.

We realized that Rust's type system can help to amend certain aspects of OpenGL which make it sometimes difficult to use.
We tried to emphasize important aspects of the API like: control flow or graphics pipeline and 
impose a structure for different APIs by generalizing them, as well as, using strongly typed procedures and types.
so users can focus on graphics programming and not have to worry about different OpenGL obscurities and verifying by hand that 
numerous contracts are upheld.

\chapter*{Introduction}

This study aims to demonstrate how extensive use of advanced type systems can change the way software is developed.
Advancements in type systems allow their users to express increasingly more complex invariants and contracts for their programs using logic rules, 
and enable type checkers to proof that all these invariants upheld at compile-time.

This is very desired as type systems not only offer extensive compile-time verification, but just as importantly logic based rules are substantially less susceptible to bugs
due to programming errors like regular validation code is.

For over a decade now we have now seen gradual shift toward both stronger typing and more of a functional approach to program specification in many languages.
with strong and expressive types systems as a foundation of this trend.

We chose to implement wrapper library for OpenGL using Rust due to several important factors.

Rust has: a state of the art type system comparable to Haskell's, 
is natively compiled language with very good support for C interoperability,
its ownership based model of resource lifetime management provides strong guarantees about when exactly destructor will be called,
borrow checker allows to better express how different pieces of program use different 

OpenGL is a very mature, cross-platform, well understood and widely supported graphics specification.
Its also arguably the best API for beginners in hardware accelerated graphics programming.
With all that being said its also quite dated, and its sequential model of execution does accurately represent the realities of graphics hardware.

We realized that Rust's type system can help to amend certain aspects of OpenGL which make it sometimes difficult to use.
We tried to emphasize important aspects of the API like: control flow or graphics pipeline and 
impose a structure for different APIs by generalizing them, as well as, using strongly typed procedures and types.
so users can focus on graphics programming and not have to worry about different OpenGL obscurities and verifying by hand that 
numerous contracts are upheld.

\chapter*{Introduction}

This study aims to demonstrate how extensive use of advanced type systems can change the way software is developed.
Advancements in type systems allow their users to express increasingly more complex invariants and contracts for their programs using logic rules, 
and enable type checkers to proof that all these invariants upheld at compile-time.

This is very desired as type systems not only offer extensive compile-time verification, but just as importantly logic based rules are substantially less susceptible to bugs
due to programming errors like regular validation code is.

For over a decade now we have now seen gradual shift toward both stronger typing and more of a functional approach to program specification in many languages.
with strong and expressive types systems as a foundation of this trend.

We chose to implement wrapper library for OpenGL using Rust due to several important factors.

Rust has: a state of the art type system comparable to Haskell's, 
is natively compiled language with very good support for C interoperability,
its ownership based model of resource lifetime management provides strong guarantees about when exactly destructor will be called,
borrow checker allows to better express how different pieces of program use different 

OpenGL is a very mature, cross-platform, well understood and widely supported graphics specification.
Its also arguably the best API for beginners in hardware accelerated graphics programming.
With all that being said its also quite dated, and its sequential model of execution does accurately represent the realities of graphics hardware.

We realized that Rust's type system can help to amend certain aspects of OpenGL which make it sometimes difficult to use.
We tried to emphasize important aspects of the API like: control flow or graphics pipeline and 
impose a structure for different APIs by generalizing them, as well as, using strongly typed procedures and types.
so users can focus on graphics programming and not have to worry about different OpenGL obscurities and verifying by hand that 
numerous contracts are upheld.


\bibliographystyle{plain}
\bibliography{bibliography}

\end{document}
