\chapter{OpenGL wrapper library}

In this chapter, we demonstrate how Rust's type system can be harnessed to create a safe wrapper library for modern OpenGL, specifically targeting version 4.6. Our goal is to cover the most essential components of the OpenGL specification and staying as close to the original spec as possible. 
In many cases, we implement a minimal subset of functionality to demonstrate that, once a specific feature is in place, it can be readily extended to encompass a broader scope of the API.

Besides the wrapper library the purpose of this study was to identify common patterns that arise during type driven design.

\section*{Overview}

The resulting library was named \textit{GPU bulwark} since it provides strong foundations for safer programming on the GPU, and could easily be extended to other GPU programming APIs.

Library at its root is logically divided into two halves: (1) main OpenGL wrapper and (2) general-purpose auxiliary modules which contain implementations of various patterns we have recognized.
%
\section{External dependencies}
%
Our library utilizes several publicly available crates from crates.io, we will briefly discuss their purposes below:
\begin{itemize}
    \item \texttt{gl} - generates raw OpenGL bindings for Rust using build script. Additionally, it exposes a single function that loads function pointers using the provided routine. 
    These bindings use C types and need to be invoked in \texttt{unsafe} context.
    \item \texttt{derive\_move} - is a procedural macro crate that expands \texttt{derive} to support more built-in traits. It significantly reduces code boilerplate.
    \item \texttt{concat\_idents} - provides singular procedural macro that allows to concatenate identifiers akin to C's \texttt{\#\#} operator. We utilize this macro for identifier generation for certain OpenGL names that strictly follow a naming convection. This yet again helps to reduce boilerplate, makes code more succinct and minimizes risk of typos.
    \item \texttt{nalgebra} and \texttt{nalgebra-glm} define algorithms and types for linear algebra computations. They are not used directly in our library for their functionality but rather for optional integration with \texttt{gpu-bulwark}.
\end{itemize}
% FIXME: replace this 
Remaining packages are imported for use in examples only.
%
\begin{itemize}
    \item \texttt{thiserror} and \texttt{anyhow} - very popular crates that make error handling more ergonomic.
    \item \texttt{raw-window-handle}, \texttt{glutin} and \texttt{winit} allow for cross platform window creation and OpenGL context initialization.
\end{itemize}

\section{Auxiliary modules and crates}

All general purpose design patterns we encountered during development are implemented in these modules.
% FIXME: relate our patterns to classical Gang of Four patterns

\section{Identified design patterns}

In our exploration we found that patterns which tend to emerge during programming with types can be broadly divided into two categories: (1) compensation for language limitations (2) validation of program structure at compile-time (CT).

\subsection{Compensation for language limitations}

Rust is in continuous development. 
Some features have been work-in-progress for over years and are still nowhere near completion. 
Others have seen minimal-viable-product releases, and some are merely the subject of wishful thinking and speculation. 
Features we found useful in type-based design fall into all of these categories. 
Most of them can be emulated with varying levels of complexity and user experience degradation.

Stemming from often contrived usage of type system and different language features
resulting error messages are very verbose and difficult to interpret.

\subsubsection{Variadic Generics}

\paragraph{Problem}

It is common practice among programming language developers to support variadic function arguments - functions which can accept arbitrarily many arguments.
This capability is a major syntactic convenience and serves as a tool for more complex abstractions.

It is substantially less common to support variadic type parameters in generic types; in fact, Rust has no such language feature.
One highly desirable use case for such variadic generics was identified: non homogenous collections. 

\paragraph{Solution}

Rust has an excellent support for recursive types thanks to the \texttt{impl} syntax, and since lists can be defined recursively we derived a variadic generics emulation scheme from that.
We used type level recursion on binary tuples (recursive step) and unit type as \texttt{Nil}. We call such recursive type list a \textbf{HList}.
HLists can be wound to the left: first tuple component contains n - 1 elements and second the n'th type or to the right in reverse order.
These two schemes are equivalent in terms of functionality, but differ in terms of potential user experience.
In our use case appending new types to the end of a HList was by far the most common use case, and as such we almost exclusively use left wound HLists (LHLists).

Implementation of functionality for HLists needs to mirror their abstract and recursive nature. This can be achieved using two \texttt{impl}s, 
one for unit in the base case and another recursive for a the binary tuple for recursive step.

These homogenous collections have been implemented as an independent module called \texttt{hlist} which can be found in the root of our crate.
 
\paragraph{Use case}

In \texttt{gpu-bulwark} HLists are used every time variable-length user configuration is required, most notably to represent shader inputs, outputs, used uniforms or external resources like textures.
Almost always we create a facade marker trait which joins together predefined pieces of functionality from \texttt{hlist} module
and adds specialized requirements for HList member types in order to prohibit creation of invalid type list.

\subsection{Const generics in const expressions}

\paragraph{Problem}

Const generics fall into the category of partially implemented features. Const generic types depend on a value of limited subset of types, most notably numeric types, bool and unit.
This feature, being in its early stages, has a significant limitation: const parameters must be literals or expressions using only literals. 
Const parameters cannot be used in any type level const expressions, they have to be used directly. 
As a result, we cannot perform arbitrary compile-time (CT) computations on these parameters for purposes of verification.

\paragraph{Solution}

However, there is one exception to that limitation: associated constants. Associated constants can have their values computed using CT \texttt{const fn}s and themselves be used in such computations as parameters.
These functions can panic with static error message (no formatting) and may cause compilation errors based on programable logic.
As a consequence, different limitation was imposed: associated constants cannot be used as const parameters in types, they can only be used as values in code.

\paragraph{Use case}

Due to the lack of negative reasoning, as of yet, in Rust compiler we cannot express type inequality.
The only viable solution would be to a blanket impl stating that two types are different if they are not the same type; since such a blanket would apply to user defined types as well.

CT validation that types are all different is required to assert that glsl variable layout locations do not overlap.
In certain scenarios when layout components are used this overlap may be valid; we ignored it in this work because it can be easily taken into account in future releases.

We use associated constants and conditionally panicking \texttt{const} function to check that location ranges do not overlap.

\subsection{Effect system}

\paragraph{Problem}

First class effect system is a non-existent feature that would be of immense value in the context of an OpenGL wrapper implementation.
Ability to type check function invocation context in OpenGL would be especially useful as we could encode presence of appropriate object binding using an effect.

\paragraph{Solution}

We instead were forced to opt for more error prone and verbose approach. 
Objects like textures or buffers can produce binder objects which in their constructor binds, and in their destructors unbinds, object from appropriate binding point.
This lets us control context bindings using lexical scope, but does not in any way prevent distinct objects with the same target from overriding the global binding.

\paragraph{Use Case}

As already mentioned above, effect system would greatly improve the handling of context bindings in terms of statically verifiable correctness, as well as, user and developer experience.

\subsection{Application of existing features}

\subsection{Markers}

\paragraph{Problem}

Enumeration types are a core component of almost all currently used programming languages. In recent years, many languages have even gained the ability to store variable-size data in their dynamic enum variants.
Such enums provide simple mechanism for statically typed polymorphism with dynamic variants. 
However, sometimes this dynamic-ness of enums is a hurdle causing constant match or switch statements to pollute the code base, producing clutter and boilerplate.
Sometime one simply wishes to encode static configuration based on a closed set of possible values.

\paragraph{Solution}

Markers are traits and types which don't provide any runtime behavior, but rather exist for purposes of conveying information and constrains on a type level.
Marker traits provide no useful functionality, but rather serve to impose relations and logical division on types.

Marker types don't hold any data and as such don't exist at runtime (they occupy zero bytes and are formally called Zero Sized Types - ZST).
It is possible for marker types to have type parameters by using special compiler intrinsic datatype \texttt{PhantomData}; which binds parameters, but does not hold any value.

Marker traits along with marker types can be used as:
\begin{itemize}
    \item compile-time enums -- by limiting access to a marker using item visibility qualifiers, we strictly control what types implement given functionality.
    \item marker trait based relations -- we can express relations between types and make unsound parameter combinations a compile-time error.
    \item typing external resources -- by using \texttt{PhantomData} we can attach type information to otherwise untyped parts of an API.
\end{itemize}

\paragraph{Use Case}

We make heavy use of markers to implement entirety of glsl module which consists almost exclusively of ZSTs for purposes of modelling shader \texttt{in}, \texttt{out} and \texttt{uniform} variables.
Types representing these variables aggregated into hlists are specified by the user with help of GLSL DSL implemented using lightweight declarative macros.

Marker traits in miscellaneous \texttt{\_::valid} modules define relations between valid combinations of data types. Buffer in raw OpenGL, due to C's lack of generics, has its buffer populated using \texttt{*void} 
and the documentation enumerates valid types. To make things worse validity of data types changes depending on what's the buffer's target.
It is illegal for index buffer to contain anything other than unsigned integers, pixel buffers can contain almost everything and vertex buffers, yet again, can contain only specific combinations of data.
By associating a phantom type with a Buffer and using marker trait based validation relations on uploaded data we solve both of these issues.

This methodology can be extended to form \textbf{many modes} pattern, in which one uses marker types that implement trait 
containing generic associated types (GAT) to control behavior in more complex fashion than using non-generic associated types.

\paragraph{many modes}

We use many modes in \texttt{Variable<S, L, T, Store>} to abstract over kind of storage used for variable's type member - \texttt{Phantom} or \texttt{Inline}.
\texttt{Phantom} uses \texttt{PhantomData} as its associated type and effectively discards value and \texttt{Inline} keeps it as is.

\subsection{Subtyping}

\paragraph{Problem}

Subtyping or inheritance is a very common concepts in object oriented programming. 
In these languages, one can create a type and inherit from it to produce more specialized version of the original type - a subtype.
Subtype extends functionality of base type and can seamlessly (without clients knowledge) delegate all base type's method calls to the base type, and can be used in all places the base type can.

\paragraph{Solution}

Using generic type and automatic dereferencing via \texttt{Deref} we can emulate the relation of subtyping.
The base type has a generic parameter which corresponds to any potential subtype, and implements \texttt{Deref} and \texttt{DerefMut} targeting that subtype.
Subtype is obtained by defining a type alias for the generic base type with concrete subtype state specified.
Subtype-specific functionality can now be specified using an \texttt{impl} block on this alias. 
Such an \texttt{impl} would be coherent since all other subtypes have different nominal types representing their states and have full access to both base type's and subtype's states.

\paragraph{Usae Case}

We applied this emulation of inheritance to model OpenGL objects. \texttt{ObjectBase<T>} contains base state and functionality and \texttt{T} provides implementation for subtype specific allocation, deallocation and context binding.
A scheme very similar to template method pattern.

\subsection{Type State}

Type state is very powerful pattern that takes advantage of how rust understands generic types and allows for tracking runtime capabilities at compile-time.
For the duration of this subsection it is worth to think about generic types as of type constructors which can be partially applied to create another constructor or fully applied to produce a type.

Recall from chapter 1 that Rust determines if \texttt{impl}s are coherent, what it means is that each associated item resolves uniquely.
Generic type with one of it's parameters supplied can have an inherent \texttt{impl} for 

\section{OpenGL wrapper}
  
\subsection*{Scope}

Our goal with this study was foremost to explore how Rust's type system can be utilized to improve static validation of OpenGL programs.
We by no means meant to cover the entirety of OpenGL functionality only the most essential aspects like shaders, programs, vertex arrays and buffers and textures.
Nevertheless, a great deal of consideration was taken before every major design decision to ensure that one could simply duplicate existing solution from a single 
API variant to all others and obtain the same level of functionality and protection.
Additionally, we tried staying as true to original OpenGL API as possible. We preserves most of the semantics, function names and parameters that made sense.
All objects don't retain any parameters, they simply forward them to appropriate GL procedure calls. 

In the end we arrived at minimal working example of a type-driven OpenGL 4.6 core wrapper that provides ability to create all the aforementioned GL objects
and program even moderately complex computer graphics in a safer fashion.

\subsection{GLSL module}

Programming in OpenGL consists of GL API and GLSL shaders. This division is at very core of our library as it's essential to correctly capture how glsl and opengl interact.

We settled on a design were shaders and pipeline configuration are the original source of truth and that's were \texttt{gpu-bulwark}'s user must start development.
Description of graphics pipeline and everything related to glsl is encompassed in the \texttt{glsl} module.
The most important component of that module is the \texttt{Variable<S, L, T, Store = md::Phantom>} type.
It represents an AST-like node for glsl variable at a type level.
It contains type parameters for 3 qualifiers we discussed in chapter 2: Type, Storage and Layout. These qualifiers are most essential from the perspective of correctness verification.
We use the \texttt{Variable}'s type parameters as follows:
\begin{itemize}
    \item type qualifier - it is used most notably in matching vertex array's attribute definitions and shader matching during program construction.
    \item storage qualifier - since variables with different storage qualifiers can share the same locations its imperative to distinguish between storage qualifiers during location overlap checking.
    \item layout qualifier - both location and binding values are used along with previous qualifiers to determine if shader inputs, outputs or uniforms are defined in a valid way.
\end{itemize}

Additionally \texttt{glsl} defines compatibility between glsl \texttt{in} types and gl attribute types.

\subsection{Shaders}

\texttt{gpu-bulwark} provides type state builders for complex objects like shaders or programs. Type state builder has a state for each type parameter we determined that object requires.
In case of shader it has one state parameter that corresponds to compilation status. The \texttt{create} method creates a shader in \texttt{Uncompiled} state.

In this state the only operation one can perform is provide shader source code. In order to attach a shader to a program, shader needs to be compiled. 
Compilation can either succeed and return shader object now in \texttt{Compiled} state or a compilation error back to the user.
Once shader has been compiled successfully one can specify what uniform variable this shader requires or precede forward by converting shader 
to either shader containing a pipeline stage entry point \texttt{main}, or to \texttt{Lib} shader for use for linking.
\texttt{Main} shader require specification of inputs or outputs using appropriate \texttt{glsl::Variable}s defined in global scope using declarative macros.

Such shaders are ready for attachment to program Objects.

\subsection{Program}

In this study we implement non-separable program objects, which means that at least vertex and fragment shaders need to be specified, as well as, that all stages have to match during linking.

\subsubsection{Builder}

Programs build using the most complex type state builder. It contains six type parameters corresponding to
\begin{itemize}
    \item target shader of most recently attached shader with entrypoint
    \item vertex input
    \item outputs of most recently configured stage
    \item program uniform definitions
    \item uniform declarations from most recently configured stage
    \item declarations of external resources used by the program
\end{itemize}

Program building is divided into two parts: (1) uniform specification and (2) pipeline configuration.


\subsubsection{Uniform Specification}

During uniform specification one can define uniforms that program contains by defining their type and providing their initial values, along with declarations of external resources that program uses.
Although both are represented by uniforms in GLSL transparent uniforms differ in their meaning from opaque types which all represent handles to resources external to the program
instead of plain old data like matrices or vectors.

\subsubsection{Pipeline Configuration}

Once uniforms and resources builder transitions to vertex shader stage where obligatory main entrypoint for vertex stage needs to be provided.
From here type state will force the user to follow valid configuration paths for the pipeline as defined in the specification, 
validating that outputs from most recently attached entrypoint shader match inputs to the newly provided one using location layout qualifiers 
on shader provided inputs and outputs, resulting in compilation error in case of a mismatch. 
This traversal always concludes with specification of the fragment stage entrypoint shader where linking can be performed and actual Program object obtained.

If program is obtained it means that pipeline is configured correctly, program is linked and ready for use. Otherwise either compilation or runtime program link error was generated.
In the current version we do not validate that representation of shader interfaces actually matches shader status, but it can be easily achieved 
by either parsing shader source code or querying shaders interfaces using OpenGL introspection API.

\subsection{Buffer}

Buffer objects heavily leverage phantom types and markers to provide type information and content type validation to otherwise untyped integers.

Buffers are created, as for all objects, with \texttt{create} function.
Once buffer is created its data store can be allocated and populated with \texttt{data} function which expects one type parameter for usage hint.

\subsection{Memory Mapping}

Buffers have an interesting feature that they can be mapped into memory and used with Direct Memory Access.
We provide this functionality using auxiliary \texttt{Mapped\_} types which borrow a buffer and are smart pointers that provide \texttt{Deref} and \texttt{DerefMut} to slices,
allowing for idiomatic store access. 
A noteworthy implementation detail is that \texttt{Mapped\_} smart pointers leverage the borrow checked to safeguard against an error condition during rendering
where draw call was issues using vertex array that sources vertex data from memory mapped buffer. 
Smart pointers hold references to buffers borrowed in turn from VAO, and since \texttt{draw\_arrays} method also requires mutable reference to VAO borrow checker will reject
a program if any buffer used as vertex source is mapped during drawing.

\subsection{Vertex Array}

Vertex Array can be created with \texttt{create}. It exposes a single method \texttt{vertex\_attrib\_pointer} which takes owenship of a Buffer bound to \texttt{BUFFER\_ARRAY}
and stores it in VAOs internal HList of \texttt{Attribute}s.
Attributes encompass buffer, vertex format and attribute index. This is a complete set of information needed to match vertex shader input variables.

\subsubsection{Draw Calls}

There are three variants of regular \texttt{draw\_arrays}. First is designed for programs with no inputs, outputs, uniforms and resources -- ones which essentially have the entire scene
encoded in shader sources code. 
Second expects vertex array object which it binds and draws as many triangles as vao specifies. Types of attributes are checked for compatibility against program inputs defined using glsl variables.
Finally the third version, the most general, expects VAO and handles to texture bindings which are matched against external resources program uses. 

\subsection{Textures}

Textures have the broadest API of all the OpenGL objects.
Textures consist of three components: (1) sampling parameters, (2) texture parameters and (3) texture images, but only the storage benefits 
from strong typing due somewhat complex allocation and wide range of rules regarding valid parameter combinations which all can be easily expressed and checked at compile-time.

Textures support mipmaps in order better antialias textures during sampling. Mipmaps are a sequence of images generated from original image by halving its dimensions until they all reach 1px.

\subsubsection{Storage}

When texture is created its storage kind (immutable, mutable, buffer) and textures dimensionality (1D, 2D or 3D) are defined by the function name.

Textures can have 3 types of their storage
\begin{itemize}
    \item mutable texture owned - allocated using \texttt{TexImage*} family of functions. 
        These are the earliest form of textures. Storage can be later reallocated, hence mutability. 
        Mipmaps need to be manually allocated and uploaded which is quite error prone.
        Pixel data for mutable storage can be specified directly using 
        \texttt{TexImage*} \texttt{data} parameter of a Buffer bound to \texttt{PIXEL\_UNPACK\_BUFFER}.
        Data may also be modified using \texttt{TexSubImage*} functions.
    \item immutable texture owned - allocated using \texttt{TexStorage*} family of functions. 
        This is the most recently added kind of texture backing storage. 
        Once texture with this storage is allocated it cannot be reallocated. 
        Major benefit of using immutable textures is that mipmaps are allocated automatically. 
        Storage immutability allows to create views into the texture providing 
        an opportunity for better memory conservation, and substantially simplifies work for the driver.
        Pixel data for these kinds of storage must be supplied using \texttt{TexSubImage*} functions.
    \item buffer backed texture - memory and content of texture come from buffer bound to \texttt{BUFFER\_TEXTURE},
        and similarly such a texture must be also bound to \texttt{BUFFER\_TEXTURE}.
\end{itemize}

In our work we focussed on immutable storage textures due to their automatic mipmap allocation which frees us from 
implements compile time mipmap completeness validation.

\subsubsection{Image Format}

Image formats (texture's internal format) are implemented in \texttt{texture::image} module.
They describe memory layout pixels in GPU's memory and their interpretation.
We focused on implementing sized formats as they specify precisely component count and size 
and such precision lends itself well to type level validation.
They are represented by a \texttt{Format<Components, ComponentType, Interpretation>}.
Besides component count and type, image formats have one additional parameter: \texttt{Interpretation},
it symbolizes how shader should interpret these values: either integers or floats.

\subsubsection{Pixels}

Once texture is allocated it needs to be initialized with pixels, by means of a pixel transfer operation.
The simplest way to transfer pixel data is by using \texttt{TexSubImage*} functions.
Ranges along each texture dimension have to be specified where pixels will be substituted,
along with pixel type and format parameters both of which we encoded in types and validate them 
against internal image format of the targeted texture.
Format parameter has dual meaning it describes: (1) what texture channels to target with current pixel transfer
(which indirectly determines number of components in provided data) with formats like \texttt{RED}, \texttt{GREEN}, \texttt{BLUE},
\texttt{BGRA} or \texttt{BGR}, and how pixels are to be interpreted during texture sampling - as integers or floats - which needs to match
texture image's \texttt{Interpretation} parameter.

Pixel transfer operations are implemented in such a way that type inference treats internal format's types
as valid ones and will validate pixels that user is trying to transfer against that and report a compilation error
if pixels are not compatible.

\section{Examples}

Examples demonstrating usage of \texttt{gpu-bulwark} are provided in \texttt{examples} directory, located in the root of the project.
There are four samples which increase in complexity as follows:
\begin{itemize}
    \item \texttt{hello\_triangle} - the simplest example that shows the most basic rendering scenario where program is hard coded to produce a white
        triangle on the screen
    \item \texttt{hello\_vertices} - this sample demonstrates basic vertex attribute configuration using a vertex buffers, 
        vertex array and program input and VAO attribute matching. 
        Two vertex attributes are used: one for vertex positions and another for vertex color.
        This sample is interactive, user can use A, S and D keys to modify color values stored in buffers 
        which is implemented using buffer memory mapping and using buffers via normal rust slices.
    \item \texttt{hello\_uniforms} - sample that shows how more complex programs which use uniforms may be constructed.
        Program implements an interactive 3D camera which can be controlled with mouse and keyboard to navigate in 3D space and
        view the scene: a single rainbow triangle. This interactivity is produced by using two uniform variables 

\end{itemize}