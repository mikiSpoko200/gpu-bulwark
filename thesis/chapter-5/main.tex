\chapter{OpenGL wrapper library}

In this chapter, we demonstrate how Rust's type system can be harnessed to create a safe wrapper library for modern OpenGL, specifically targeting version 4.6. Our goal is to cover the most essential components of the OpenGL specification and staying as close to the original spec as possible. 
In many cases, we implement a minimal subset of functionality to demonstrate that, once a specific feature is in place, it can be readily extended to encompass a broader scope of the API.

Besides the wrapper library the purpose of this study was to identify common patterns that arise during type driven design.

\section*{Overview}

The resulting library was named \textit{GPU bulwark} since it provides strong foundations for safer programming on the GPU, and could easily be extended to other GPU programming APIs.

Library at it's root is logically divided into two halves: (1) main OpenGL wrapper and (2) general-purpose auxiliary modules which contain implementations of various patterns we have recognized.
%
\section{External dependencies}
%
Our library utilizes several publicly available crates from crates.io, we will briefly discuss their purposes below:
\begin{itemize}
    \item \texttt{gl} - generates raw OpenGL bindings for Rust using build script. Additionally, it exposes a single function that loads function pointers using the provided routine. 
    These bindings use C types and need to be invoked in \texttt{unsafe} context.
    \item \texttt{derive\_move} - is a procedural macro crate that expands \texttt{derive} to support more built-in traits. It significantly reduces code boilerplate.
    \item \texttt{concat\_idents} - provides singular procedural macro that allows to concatenate identifiers akin to C's \texttt{\#\#} operator. We utilize this macro for identifier generation for certain OpenGL names that strictly follow a naming convection. This yet again helps to reduce boilerplate, makes code more succinct and minimizes risk of typos.
    \item \texttt{nalgebra} and \texttt{nalgebra-glm} define algorithms and types for linear algebra computations. They are not used directly in our library for their functionality but rather for optional integration with \texttt{gpu-bulwark}.
\end{itemize}
% FIXME: replace this 
Remaining packages are imported for use in examples only.
%
\begin{itemize}
    \item \texttt{thiserror} and \texttt{anyhow} - very popular crates that make error handling more ergonomic.
    \item \texttt{raw-window-handle}, \texttt{glutin} and \texttt{winit} allow for cross platform window creation and OpenGL context initialization.
\end{itemize}

\section{Auxiliary modules and crates}

All general purpose design patterns we encountered during development are implemented in these modules.
% FIXME: relate our patterns to classical Gang of Four patterns

\section{OpenGL wrapper}
