\chapter{Basics of Computer Graphics}

%NOTE: Zastanawiam się czy ten rozdział jest potrzebny - i tak będę musiał opisać potok w rozdziale o OpenGL'u.

\section*{Introduction}

The task of 3D computer graphics is to generate a 2D array of discrete color values from geometric representation of a scene. This process can be divided into two halves:
\begin{enumerate}
  \item Geometry processing -- manipulation of geometric data that
  \item Pixel processing -- 
\end{enumerate}

Between these two halves the rasterization process happens which transforms geometric data into set of discrete values 

\subsection{Programmable graphics pipeline}

%FIXME: czy jest sens w ogóle pisać o historii sprzętu? Na pewnym etapie myślałem, że potrzebuję mówić o programowalnych kartach ale teraz nie widzę już takiej potrzeby.
Release of Nvidia's GeForce 256 was a significant milestone for 3D graphics hardware. It was the first graphics accelerator that was capable of on-chip Transform and Light calculations which allowed it to process all of the pipeline without interacting with the CPU.

As power of, now called Graphics Processing Units (GPUs), grew it 
